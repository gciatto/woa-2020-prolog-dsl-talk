% !TeX spellcheck = en_GB
%
\documentclass[presentation]{beamer}
\mode<presentation>{\usetheme{AMSCesenaPurpleAndGold}}
\setbeamertemplate{bibliography item}{\insertbiblabel}
%%%%%%%%%%%%%%%%%%%%%%%%%%%%%%%%%%%%%%%%%%%%%%%%%%%%%%%%%%%%%%%%%%%%%%%%%%%%%%%%
\usepackage[english]{babel}
\usepackage[utf8]{inputenc}
%
\usepackage{woa-2020-prolog-dsl-talk}
%%%%%%%%%%%%%%%%%%%%%%%%%%%%%%%%%%%%%%%%%%%%%%%%%%%%%%%%%%%%%%%%%%%%%%%%%%%%%%%%
\title[\twopkt{}: LP in Kotlin]{
    \twopkt{}: logic programming with objects \& functions in Kotlin
}
%
% \subtitle{Extended Abstract}
%
% same authors order of the presented paper
\author[Ciatto et al.]{
    Giovanni Ciatto$^*$
    \and
    Roberta Calegari$^\circ$
    \and
    Enrico Siboni$^\dagger$
    \\
    Enrico Denti$^\ddagger$
    \and
    Andrea Omicini$^\star$
}
%
\institute[UniBo, HES-SO]{
    $^{*\ddagger\star}$Dipartimento di Informatica -- Scienza e Ingegneria (DISI)
    \\
    $^{\circ}$Alma Mater Research Institute for Human-Centered Artificial Intelligence
    \\
    \textsc{Alma Mater Studiorum}---Università di Bologna, Italy
    \\
    \texttt{\{giovanni.ciatto, roberta.calegari, enrico.denti, andrea.omicini\}@unibo.it}
    \\\medskip
    $^\dagger$University of Applied Sciences and Arts of Western Switzerland (HES-SO)
    \\
    \texttt{enrico.siboni@hevs.ch}
}
%
\date[WOA, 2020]{
	21$^{st}$ Workshop ``From Objects to Agents'' (WOA)
	\\
	Sept. 16, 2020, Bologna, Italy
}
%%%%%%%%%%%%%%%%%%%%%%%%%%%%%%%%%%%%%%%%%%%%%%%%%%%%%%%%%%%%%%%%%%%%%%%%%%%%%%%%
\begin{document}
%%%%%%%%%%%%%%%%%%%%%%%%%%%%%%%%%%%%%%%%%%%%%%%%%%%%%%%%%%%%%%%%%%%%%%%%%%%%%%%%

%\\\\\\\\\\\\\\\\\\\\\
\frame{\titlepage}
%\\\\\\\\\\\\\\\\\\\\\

%===============================================================================
\section{Motivation \& Context}
%===============================================================================

%\\\\\\\\\\\\\\\\\\\\\
\begin{frame}[c]{Context}

    \begin{block}{AI side}
        \begin{itemize}
            \item AI is shining, brighter than ever
            %
            \begin{itemize}
                \item mostly thanks to the advances in ML and sub-symbolic AI
            \end{itemize}

            \item[$\Rightarrow$] symbolic AI is gaining momentum because of XAI
            %
            \begin{itemize}
                \item[!] hybrid solution mixing logic \& data-driven AI are flourishing \ccite{xaisurvey-ia2020}
            \end{itemize}
        \end{itemize}
    \end{block}

    \begin{block}{MAS side}
        The MAS community is eager for logic-based technologies \ccite{lptech4mas-jaamas2020}
        %
        \begin{itemize}
            \item to support agents' knowledge representation, reasoning, or execution

            \item or to prove MAS properties

            \item[!] despite few mature tech exist, and even fewer are actively maintained
        \end{itemize}
    \end{block}
\end{frame}
%\\\\\\\\\\\\\\\\\\\\\

%\\\\\\\\\\\\\\\\\\\\\
\begin{frame}[c]{Motivation}

    \begin{alertblock}{The problem with logic-based technologies}
        There is technological barrier slowing
        %
        \begin{itemize}
            \item the adoption of logic programming (LP) as paradigm
            \item the exploitation of logic-based technologies
        \end{itemize}
        %
        while programming \emph{in the large}
    \end{alertblock}

    \vfill

    \begin{itemize}
        \item $\overbrace{\text{mainstream programming languages}}^\text{\itshape e.g. Scala, Kotlin, Python, C\#}$ are blending several paradigms
        %
        \begin{itemize}
            \item[e.g.] imperative, object-oriented (OOP), and functional programming (FP)
            \item except LP!
        \end{itemize}

        \vfill

        \item $\underbrace{\text{mainstream platforms}}_\text{\itshape e.g. JVM, .NET, JS, Python}$ are poorly interoperable with logic-based tech.
    \end{itemize}

\end{frame}
%\\\\\\\\\\\\\\\\\\\\\

%\\\\\\\\\\\\\\\\\\\\\
\begin{frame}{Motivating example -- SWI-Prolog's FLI for Java}

    \begin{itemize}
        \item Many Prolog implementors rely on Foreign Language Interfaces (FLI)
        %
        \begin{itemize}
            \item (mostly targetting Java, or C)
        \end{itemize}

        \vfill

        \item For instance, SWI-Prolog comes with a FLI for Java:
        %
        \inputminted{java}{code/SwiFliExample.java}

        \vfill

        \item[$\rightarrow$] No paradigm harmonization between Prolog and $\underbrace{\text{the hosting language}}_\text{\itshape i.e. Java}$
    \end{itemize}

\end{frame}
%\\\\\\\\\\\\\\\\\\\\\

%\\\\\\\\\\\\\\\\\\\\\
\begin{frame}{Contribution of the paper}

\begin{itemize}
    \item Show that OOP, FP, and LP can be blended into a single language
    \item contribution 2
\end{itemize}

\end{frame}
%\\\\\\\\\\\\\\\\\\\\\

%===============================================================================
\section{Theory / modelling / design}
%===============================================================================

%\\\\\\\\\\\\\\\\\\\\\
\begin{frame}%[allowframebreaks]
\frametitle{Theory / modelling / design}

    Provide 2-3 slides discussing the Theory / modelling / design

\end{frame}
%\\\\\\\\\\\\\\\\\\\\\

\section{Case study / Experiments / Results}

%\\\\\\\\\\\\\\\\\\\\\
\begin{frame}%[allowframebreaks]
\frametitle{Case study / Experiments / Results}

    Provide 2-3 slides discussing the Case study / Experiments / Results of the paper

\end{frame}
%\\\\\\\\\\\\\\\\\\\\\

\section{Conclusions \& future works}

%\\\\\\\\\\\\\\\\\\\\\
\begin{frame}%[allowframebreaks]
\frametitle{Conclusions \& future works}

\begin{block}{Summing up}
    Summarise the most relevant contributions of this study:
    %
    \begin{itemize}
        \item conclusion 1
        \item conclusion 2
        \item conclusion 3
    \end{itemize}
\end{block}

\begin{exampleblock}{Future works}
    Sketch some future research directions
    %
    \begin{itemize}
        \item future work 1
        \item future work 2
    \end{itemize}
\end{exampleblock}

(may be split into 2 slides)

\end{frame}
%\\\\\\\\\\\\\\\\\\\\\

%===============================================================================
\section*{}
%===============================================================================
\frame{\titlepage}

%===============================================================================
\section*{\bibname}
%===============================================================================

\setbeamertemplate{page number in head/foot}{}
%\\\\\\\\\\\\\\\\\\\\\
\begin{frame}[t,allowframebreaks,noframenumbering]\frametitle{\refname}
% \begin{frame}[c]\frametitle{\refname}
	\footnotesize
%	\scriptsize
    \bibliographystyle{plain}
	\bibliography{woa-2020-prolog-dsl-talk}
\end{frame}
%\\\\\\\\\\\\\\\\\\\\\

%%%%%%%%%%%%%%%%%%%%%%%%%%%%%%%%%%%%%%%%%%%%%%%%%%%%%%%%%%%%%%%%%%%%%%%%%%%%%%%%
\end{document}
%%%%%%%%%%%%%%%%%%%%%%%%%%%%%%%%%%%%%%%%%%%%%%%%%%%%%%%%%%%%%%%%%%%%%%%%%%%%%%%%
